\documentclass[12pt]{article}

\usepackage{makeidx}
\usepackage{hyperref}
\usepackage{amsmath}
\usepackage{float}
\makeindex

% a todo macro                                                                  
\newcommand{\todo}[1]{\vspace{3 mm}\par \noindent {\textsc{ToDo}}\framebox{
\begin{minipage}[c]{1.0\hsize}\tt #1 \end{minipage}}\vspace{3mm}\par}

\floatstyle{ruled}
\newfloat{program}{thp}{lop}
\floatname{program}{Program}

\newfloat{struct}{thp}{lop}
\floatname{struct}{Data Structure}

\setcounter{tocdepth}{4}


\begin{document}
\title{libTF Manual}
\author{Tully Foote\\
\href{mailto:tfoote@willowgarage.com}{\texttt{tfoote at willowgarage.com}}}
\date{\today}
\maketitle

\tableofcontents
\pagebreak

\section{Example Usage}
\subsection{Library}
%\begin{program}[H]
\begin{verbatim}
#include ``libTF/libTF.h''
#include <sys/time.h>

using namespace std;

int main(void)
{
  double dx,dy,dz,dyaw,dp,dr;
  TransformReference mTR;
  
  //Temporary Variables
  dx = dy= dz = 0;
  dyaw = dp = dr = 0.1;
  
  timeval temp_time_struct;
  gettimeofday(&temp_time_struct,NULL);
  unsigned long long atime = temp_time_struct.tv_sec * 1000000000ULL + (unsigned long long)temp_time_struct.tv_usec * 1000ULL;

  
  //Fill in some transforms
  //  mTR.setWithEulers(10,2,1,1,1,dyaw,dp,dr,atime); //Switching out for DH par
ams below
  mTR.setWithDH(10,2,1.0,1.0,1.0,dyaw,atime);
    //mTR.setWithEulers(2,3,1-1,1,1,dyaw,dp,dr,atime-1000);
   mTR.setWithEulers(2,3,1,1,1,dyaw,dp,dr,atime-100);
   mTR.setWithEulers(2,3,1,1,1,dyaw,dp,dr,atime-50);
   mTR.setWithEulers(2,3,1,1,1,dyaw,dp,dr,atime-1000);
   //mTR.setWithEulers(2,3,1+1,1,1,dyaw,dp,dr,atime+1000);
  mTR.setWithEulers(3,5,dx,dy,dz,dyaw,dp,dr,atime);
  mTR.setWithEulers(5,1,dx,dy,dz,dyaw,dp,dr,atime);
  mTR.setWithEulers(6,5,dx,dy,dz,dyaw,dp,dr,atime);
  mTR.setWithEulers(6,5,dx,dy,dz,dyaw,dp,dr,atime);
  mTR.setWithEulers(7,6,1,1,1,dyaw,dp,dr,atime);
  mTR.setWithDH(8,7,1.0,1.0,1.0,dyaw,atime);
  //mTR.setWithEulers(8,7,1,1,1,dyaw,dp,dr,atime); //Switching out for DH params
 above
  
  
  //Demonstrate InvalidFrame LookupException
  try
    {
      std::cout<< mTR.viewChain(10,9);
    }
  catch (TransformReference::LookupException &ex)
    {
      std::cout << ``Caught `` << ex.what()<<std::endl;
    }
  
  
  // See the list of transforms to get between the frames
  std::cout<<''Viewing (10,8):''<<std::endl;  
  std::cout << mTR.viewChain(10,8);
  
  
  //See the resultant transform
  std::cout <<''Calling getMatrix(10,8)''<<std::endl;
  NEWMAT::Matrix mat = mTR.getMatrix(10,8,atime);  
  std::cout << ``Result of getMatrix(10,8,atime):'' << std::endl << mat<< std::end
l;
  
  //Break the graph, making it loop and demonstrate catching MaxDepthException
  mTR.setWithEulers(6,7,dx,dy,dz,dyaw,dp,dr,atime);
  
  try {
    std::cout<<mTR.viewChain(10,8);
  }
  catch (TransformReference::MaxDepthException &ex)
    {
      std::cout <<''caught loop in graph''<<std::endl;
    }
  
  //Break the graph, making it disconnected, and demonstrate catching Connectivi
tyException
  mTR.setWithEulers(6,0,dx,dy,dz,dyaw,dp,dr,atime);

  try {
    std::cout<<mTR.viewChain(10,8);
  }
  catch (TransformReference::ConnectivityException &ex)
    {
      std::cout <<''caught unconnected frame''<<std::endl;
    }  
  return 0;
};

\end{verbatim}
%\end{program}

\subsection{ROS Implementation}

\subsubsection{Example Broadcaster}
\begin{verbatim}
#include "rosTF/rosTF.h"

class testServer : public ros::node
{
public:
  //constructor
  testServer() : ros::node("server"),count(2){
    pTFServer = new rosTFServer(*this);
  };

  //A pointer to the rosTFServer class
  rosTFServer * pTFServer;


  // A function to call to send data periodically
  void test () {
    pTFServer->sendEuler(5,count++,1,1,1,1,1,1,100000,100000);
    pTFServer->sendDH(5,count++,1,1,1,1,100000,100000);
    pTFServer->sendQuaternion(5,count++,1,1,1,1,1,1,1,100000,100000);
  };

private:
  int count;

};

int main(int argc, char ** argv)
{
  //Initialize ROS
  ros::init(argc, argv);

  //Construct/initialize the server
  testServer myTestServer;
  
  while(myTestServer.ok())
    {
      //Send some data
      myTestServer.test();
      sleep(1);
    }

  return 0;
};
\end{verbatim}

\subsubsection{Example Client}
\begin{verbatim}
#include "rosTF/rosTF.h"

class testListener : public ros::node
{
public:
  //constructor with name
  testListener() : ros::node("client") {
    pClient = new rosTFClient(*this);
  };

  //A pointer to the client library object  
  rosTFClient * pClient;

};


int main(int argc, char ** argv)
{
  //Initialize ROS
  ros::init(argc, argv);

  //Instantiate a local listener
  testListener testListener;
  
  //Nothing needs to be done except wait for a quit
  //The callbacks withing the listener class 
  //will take care of everything
  while(testListener.ok())
    {
      sleep(1);
    }

  return 0;
};
\end{verbatim}

\bibliographystyle{plain}
\bibliography{libTF_Manual}

\printindex
\end{document}

